%!TEX root = jir2018aspects.tex
\section{Summary and Conclusions} \label{sec:conclusion}

In this paper, we investigated the effects of diversifying search results when searchers undertook complex search tasks, requiring one to learn about different aspects of a topic. We inferred a number of hypotheses based upon Information Foraging Theory, in which diversification would lead to searchers examining fewer documents per query, and subsequently issuing more queries. We tested our hypotheses by conducting a within-subjects user study, using \emph{(i)} a non-diversified system; versus \emph{(ii)} a diversified system, when the search task was either: \emph{(a)} ad-hoc; or \emph{(b)} aspectual.

Our findings lend evidence to support our hypotheses broadly; however, our results were not statistically significant. This was despite the fact that there were significant differences in the two systems performance, i.e. the diversified system returned a ranked list of results with a greater number of documents containing new, unseen entities. Clearly, bigger differences need to be present before participants can subjectively report whether they had a different experience, or which one they preferred -- as post-task and system questions revealed no difference. However, in terms of performance, we found that participants on the diversified system did perform better -- more relevant documents were found, and more new entities were found -- suggesting they found out more about the topics on the diversified system. They also inspected few non-relevant documents. After conducting a post-hoc analysis, we showed that the hypotheses we posited given IFT were sound, but revised our expectations on how participants would behave when using the diversified system. That is, they would examine more documents per query, and thus issue fewer queries when undertaking the aspectual retrieval task, as opposed to there being no difference in performance. Again, we see a trend to support the hypothesis. Encouragingly, our application of Information Foraging Theory, before and after the study, led to new insights into how behaviours are affected under the different conditions -- and is a useful tool in developing, motivating and analysing search performance and behaviours. Counter to our intuition about how we \emph{believed} people would behave in these conditions, the theory provided more informed and accurate hypotheses.

In past work, mainly interface based solutions were studied -- where few significant differences in behaviour were found compared to a standard interface. Disappointingly, we also find that an algorithmic solution has very little influence either, though there were trends which indicated that diversifying the results does lead to better performance, greater awareness of the topic (even when not specifically instructed, i.e. \textit{find relevant only}), and fewer examinations of non-relevant items. Thus, we suggest that diversification should be employed more widely -- in particular in the context of news search -- where bias is an issue, and diversification algorithms can present a broader overview of the aspects within a topic. 


%Despite this, however, the subjects of this study found it difficult to discern the difference between the two systems, suggesting that searchers don't really notice whether a set of results have been diversified or not. Indeed, regardless of whether or not results were diversified, performance of the subjects remained similar across the two systems. Subjects did identify more entities using the diversified system -- but still performed well without it. Searchers are able to adapt their behaviour to extract a decent level of performance out, regardless of the search task and system that they are using.

