%!TEX root = jir2018aspects.tex
\section{Summary and Conclusions} \label{sec:conclusion}

In this paper, we investigated the effects of diversifying search results when searchers undertook complex search tasks, where one was required to learn about different aspects of a topic. We inferred a number of hypotheses based upon IFT in which diversification would lead to searchers examining fewer documents per query and subsequently issuing more queries. We tested our hypotheses by conducting a within-subjects user study, using \emph{(i)} a non-diversified system; versus \emph{(ii)} a diversified system, when the retrieval task was either: \emph{(a)} ad-hoc; or \emph{(b)} aspectual in nature.

Our findings lend evidence to support the IFT hypotheses broadly. However, we only observed statistically significant differences across a subset of behavioural and temporal measures. This was despite the fact that there were significant differences in the performance of the two systems -- the diversified system was able to, on average, return a ranked list of results with a greater number of documents containing new, unseen entities. This finding is in line with past work which found that interface-based interventions seemingly had little influence on search performance and search behaviours. Clearly, bigger differences need to be present -- or larger sample sizes are required -- to determine if the difference between systems over all examined indicators is significant. Despite these results, there were a number of clear trends.

When performing the aspectual task on the diversified system \textbf{\emph{D}} (in contrast to the Non-Diversified System):  participants examined fewer documents per query (3 vs. 3.7 documents/query), issued slightly fewer queries (5.9 vs. 5.2 queries), and didn't go to as great a depth when examining SERPs (depths of 12.8 vs 15.7). Taken together this resulted in a lower probability of clicking ( $P(C)$ = 0.16 vs 0.21, which was significantly different) and interestingly a lower probability of clicking on non-relevant ($P(C|N)$ = 0.13 vs. 0.18, which was also significantly different). While participants spent a similar amount of time searching on both systems, participants on the diversified system spent slightly more time examining each document (16 seconds vs. 13 seconds), and more time in total examining documents (163 seconds vs. 145 seconds) - suggesting that more effort was directed to assessing rather than searching. However, participants found significantly more entities (7.2 vs. 4.3 entities) and found more documents that contained new/different entities (3.2 vs 2.4). Both of these findings were statistically significant. This shows that the diversification algorithm led to a greater awareness of the topics and provided participants with greater coverage of the topic - which suggests that participants were able to learn more about the topic, and were exposed to less bias.

When performing the ad-hoc task over the diversified system \textbf{\emph{D}} (in contrast to the non-diversified system \textbf{\emph{ND}}): participants examined more documents per query (3.48 vs. 3.23 documents/query), issued slightly more queries (4.96 vs. 5.20 queries), and examined content to greater depths presented on SERPs (depths of 16.2 vs. 13.9). Again, this meant that the probability of clicking was lower on the diversified system (0.16 vs. 0.20); this was significantly so. Participants spent similar amounts of time searching on both systems. However, unlike on the aspectual tasks, participants spent less time examining potentially relevant documents on system \textbf{\emph{ND}} (13.7 vs. 15.1 seconds), and they spent less time in total assessing documents (139.6 vs, 152.8 seconds). This suggests that less effort was directed at assessing, rather than searching. This could be possibly due to the performance of the diversified system being higher than the non-diversified system ($P@5=0.29$ vs. $0.25$, which was significantly different). Alternatively, it could be because the results returned were easier to identify as relevant as the probability of marking a document given it was relevant was higher (0.74 vs. 0.67). This suggests that participants may be more confident when using the diversified system. Although not explicitly requested in the task description, participants encountered more novel entities when using the diversified system (5.8 vs. 4.4). Participants also found more documents with new entities using the diversified system (2.6 vs. 2.0). Taken together, this suggests that participants again implicitly learn more about the topic because the diversified system surfaced content that presented a more varied view on the topic.

%More relevant documents were found, and more new entities were found, suggesting they found out more about the topics on the diversified system. They also inspected fewer non-relevant documents. 

With regards to the application of IFT, we showed that generated hypotheses were largely sound, but the empirical data prompted us to revise the hypotheses. Initially, we hypothesised that the performance and behaviour on both tasks would be similar when using the diversified system (see Fig.~\ref{fig_ift_patches}(b)). However, post-hoc analysis revealed that the performance (and subsequent behaviour) was different (see Fig.~\ref{fig_cg}(b)). Here, participants obtained higher levels gain for the ad-hoc task. Thus, under such conditions, IFT would stipulate that they would examine more documents per query (3.48 vs 3.02 documents/query) and issued fewer queries (4.9 vs. 5.9 queries) when undertaking the ad-hoc retrieval task vs. the aspectual retrieval task (as opposed to there being no difference). Encouragingly, our application of IFT (before and after the experiment) led to new insights into how behaviours are affected under different conditions. This shows that IFT is a useful tool in developing, motivating and analysing search performance and behaviours. Furthermore, counter to our intuition about how we \emph{believed} people would behave in these conditions, the theory provided \emph{more informed and accurate hypotheses} which tended to hold in practice.

This work motivates further research into complex search tasks and the impact of diversifying search results. Diversification can play an important role in improving the search experience by providing greater coverage of a topic, and mitigating potential biases that may exist in search results. One such avenue for further exploration is a per-topic analysis. As our results were presented as a mean over each experimental condition (or system, or task, as reported in the bottom table of Table~\ref{tbl_actions}), we may have missed important per-topic differences. This argument can be motivated from the large variance in the number of different entities identified for each topic (with \emph{Airport Security} having only 14 different airports, and \emph{Wildlife Extinction} possessing 168 different species of endangered animal). However, a reported low variance in the number of entities found (see the \emph{\#Entities Found} rows in Table~\ref{tbl_actions}) suggests that this may not be the case. Further work includes an investigation into how such search behaviours and performance would vary with larger sample sizes, yielding increased experimental power. An examination of how these behaviours and performance under different retrieval tasks and search contexts would also be an interesting area for future exploration. For example, \emph{would different retrieval tasks affect the perceptions of individuals and their decisions?} Finally, an examination of different diversification algorithms would provide us with a better understanding of how diversification influences search behaviours and performance.

In conclusion, we found that in terms of search behaviour: participants on the diversified system issued more queries and examined fewer documents per query when performing aspectual search tasks. Furthermore, we showed that when using the diversified system, participants were more successful in marking relevant documents, and obtained a greater awareness of the topics (i.e. identified relevant documents containing novel aspects). This was also the case even when they were not specifically instructed to do so (i.e. when performing the ad-hoc search task). These findings suggest that diversification should be employed more widely, particularly where bias is a potential issue (such as in news search). Here, diversification algorithms would the be able to present a broader overview of the aspects within a topic.   



%This study motivates further research in the influe

%Despite this, however, the subjects of this study found it difficult to discern the difference between the two systems, suggesting that searchers don't really notice whether a set of results have been diversified or not. Indeed, regardless of whether or not results were diversified, performance of the subjects remained similar across the two systems. Subjects did identify more entities using the diversified system -- but still performed well without it. Searchers are able to adapt their behaviour to extract a decent level of performance out, regardless of the search task and system that they are using.

