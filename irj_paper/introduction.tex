%!TEX root = jir2018aspects.tex
\section{Introduction} \label{sec:intro}
\emph{Interactive Information Retrieval (IIR)} is a complex (and often exploratory) process~\cite{ingwersen2005theturn} in which a searcher issues a variety of queries as a means to explore the topic space~\cite{kelly2015search_tasks}. Often, such tasks are \emph{aspectual} in nature, where an underlying goal is to find out about the different facets, dimensions or aspects of the topic. This type of task is often referred to as \emph{aspectual retrieval}. While aspectual retrieval has been heavily studied in the past (during the TREC Interactive Tracks~\cite{over2001trec}), there has been renewed interest in the search task as it represents a novel context to explore the idea of \emph{``search as learning''}~\cite{collins2017sal}. In this context, the goal of the system is to help the searcher learn about a topic~\cite{collins2017sal} -- and in doing so, the number of aspects that the searcher finds indicates how much they learned during the process~\cite{syed2017sal}. If the goal is to help people learn about a topic, then by returning results that are more diverse in nature and presenting a broader view on the topic, these changes \emph{should} help searchers learn more about the said topic. This reasoning suggests that employing \emph{diversification} will lead to an improved search and learning experience~\cite{syed2017sal}. 

While there have been numerous diversification algorithms developed and proposed over the years~\cite{carbonell1998mmr,chen2006lessismore,santos2010query_reformulations_diversification,santos2011intent,zhai2015subtopics}, the focus here has been on addressing the problem of intents, rather than how diversification affects complex search tasks, such as \emph{ad-hoc} or aspectual retrieval. In this paper, we perform one of the first investigations into the influence and impact of result diversification on search behaviour and search performance when performing different search tasks (ad-hoc or aspectual). Our focus is on understanding how behaviours -- in particular, how searching and stopping behaviours -- change under the different conditions. We ground our study by drawing upon \emph{Information Foraging Theory (IFT)}~\cite{pirolli1999ift} (see Section~\ref{sec:background}) which derives the following hypotheses regarding diversification when performing aspectual search tasks: \emph{(i)} diversification will lead to searchers examining fewer documents per query; and either \emph{(ii)} issuing more queries, or \emph{(iii)} completing the task in less time. However, these hypotheses seem to be counter to our intuition. If a system provides a more diversified set of results, then searchers \emph{should} be able to exploit the diversification of results and find more varied aspects by examining more documents for a given query -- and thus issue fewer queries. In order to explore the validity of the IFT hypotheses and test our intuitions, we designed a $2 \times 2$ within-subjects user study, where participants were tasked to learn about four different topics under the following conditions, using: \emph{(i)} a non-diversified system (\emph{BM25}); versus \emph{(ii)} a diversified system (BM25+\emph{xQuAD}~\cite{santos2010query_reformulations_diversification}), and when the search task is either: \emph{(a)} ad-hoc retrieval, where they need only to find relevant documents; or \emph{(b)} aspectual retrieval, where they need to find documents that are both relevant and different -- i.e. covering new, unseen aspects of the topic. We perform our experiments in the context of learning about a topic to write a report where participants use a standard search interface to search the \emph{TREC AQUAINT} news collection. 

%%%%%%%%%%
%\emph{Interactive Information Retrieval (IIR)} is a complex, and often exploratory, process in which the searcher undertakes various actions over the course of a search session in order to learn about a topic~\cite{ingwersen2005theturn}. During such complex search tasks, searchers issue a variety of queries as a way to explore the topic space so that they can obtain a greater awareness and understanding of the topic~\cite{kelly20XXictir}. Such tasks can been consider \emph{aspectual}, in nature, where the goal is to learn about or find out about a number of different aspects on the same topic i.e. Aspectual Retrieval. As part of the TREC Interactive Tracks, a significant amount of research was directed towards developing systems and interfaces to help users explore and retrieval various aspects of a topic, e.g. cluster based and faceted interfaces to explicitly show different aspects\cite{}, query suggestions to recommend different paths\cite{} and tiles and stacks to organize documents\cite{}. However, a disapointing conclusion from this initiative, was that there were little differences between the standard control (ten blue links) systems, and the proposed systems in terms of behaviour and performance~\cite{voorhees05trec}. 

%As work on aspectual retrieval subsidied, work related to determining the intent of a query flourished, where the goal is to diversify the results retrieved with respect to the query~\cite{rose2004understanding_user_goals} - and thus address the problem ambiguity of short impoverished queries. This lead to a series of diversification algorithms (and intent-aware evaulation measures) being proposed. However, while there have been numerous studies investigating the effectiveness of diversification algorithm (for the problem of intents, i.e., one query, many meanings), there has been little work studying how such algorithms apply in the context of aspect retrieval (i.e., one topic, many aspects). Recently, however, within the context of ``search as learning'' there has been a renewed interest in the task as it represents a learning context that involves exploratory search - where the goal of the system, in such a setting, is to help the searcher explore and learn about a topic~\cite{sal2016dagstuhl}.  In~\cite{}, they explored how people learn about particular topics when searching using a system designed to diversify the results based on the new vocabulary introduced. Their work connects the problem of aspect retrieval with the idea of diversification to motivate the present study. 

%In this paper, we conduct an investigation into how diversification (or not) affects the search behaviour and search performance of users when undertaking different complex search tasks (aspectual and ad hoc retrieval). Our focus is on understanding how searching and in particular stopping behaviours change under the different conditions, and whether there is a significant change in user search behaviours. We hypothesise that diversification will reduce the number of queries users issue, while increasing the number of documents that they examine per query. Furthemore, we anticipate that by using a diversified system, users will experience more aspects, find relevant documents that cover more aspects, and thus learn more about the topic. To explore these research questions we designed a 2x2 within-subjects user study where participants need to find out about four different topics for a news report....
%%%%%%%%%%



%We therefore in this paper report on a within-subjects user study, designed to allow us to examine the differences in searcher behaviour and performance when subjects were provided with diversified content, and when they weren't. We also considered the overall search goal when considering diversity, as the task has shown to affect search behaviour~\cite{rose2004understanding_user_goals}. The study therefore allows us to address the following two research questions. \textbf{RQ1} How does search diversification affect a user's search behaviour and performance? \textbf{RQ2} How is search behaviour and performance affected when the overall search task required diversification?

%Despite the advancements in the development of the underlying components that present diversified results to searchers, little research has been undertaken into how exactly diversification affects the search performance and behavioural aspects of searchers. 




% However, a user's information need is typically loosely defined when they arrive at a search engine. This \emph{Anomalous State of Knowledge (ASK)}~\cite{belkin1980anomalous_states} therefore typically suggests that a searcher will arrive some some degree of ambiguity of what they are looking for. Conversely, even with a well-defined information need, it may be complex, leading to further potential ambiguity. 

%\todo{lets describe the different ideas/notions of diversity/aspects/intents}

%The searcher may also have a good idea of what they are looking for; converting this need to a salient, understandable, unambiguous query may be difficult with a complex need. Indeed, many of the queries submitted to commercial web search engines today are ambiguous in nature, with several possible interpretations~\cite{sparckjones2007ambiguity} (i.e. with the query \texttt{`python'}, is the searcher looking for information on the \emph{Python} programming language, or \emph{Pythonidae} family of snakes?). Even if the query is not ambiguous, different \emph{aspects} for the particular information need may exist, with a searcher wishing to explore the different aspects/subtopics that satisfy their information need.
%\todo{this seems to be more intents rather than aspects..}


%Given the potential ambiguity that search engines are faced with, how can the designers of these systems avoid potential issues of not providing the searcher with what they want? Solutions to the problem were outlined by Santos et al.~\cite{santos2010query_reformulations_diversification}, with the \emph{safest bet} approach to \emph{diversify} the presented results to the searcher. By diversifying and presenting a series of results with potentially different aspects of the interpreted information need, hopefully one of the meaning presented will assist in satisfying the user~\cite{agrawal2009diversification}. Indeed, the concept of diversifying the results presented on the \emph{Search Engine Results Page (SERP)} has over the past two decades resulted in a number of different studies in the area within the \emph{Information Retrieval (IR)} community, with a number of different diversification algorithms \todo{(cite)}, test collections \todo{(cite)} and measures \todo{(cite)}.



%\emph{Interactive Information Retrieval (IIR)} is a complex, non-trivial process in which a searcher undertakes a number of different actions over the course of a search session~\cite{ingwersen2005theturn}. A searcher will typically arrive at a search engine with some loosely defined mental model of their underlying information need -- called an \emph{Anomalous State of Knowledge (ASK)}~\cite{belkin1980anomalous_states} -- and thus will arrive with some degree of \emph{ambiguity}. Even if their information need is well formed, it may be complex, leading to ambiguity. Indeed, many of the queries submitted to many commercial web search engines today are ambiguous in nature, with several possible interpretations~\cite{sparckjones2007ambiguity} (i.e. with the query \texttt{python}, is the searcher looking for information on the \emph{Python} programming language, or \emph{Pythonidae} family of snakes?).


% The \emph{Interactive Information Retrieval (IIR)} process is a complex, non-trivial process in which a searcher undertakes a variety of different actions during the course of a search session~\cite{ingwersen2005theturn}. Searchers typically arrive at a search engine in an \emph{Anomalous State of Knowledge (ASK)}~\cite{belkin1980anomalous_states} -- perhaps with a rough idea of what they are looking for, but may not have a sufficient mental model of the underlying information need to formulate an unambiguous query (i.e. does the query \texttt{python} denote the \emph{Python} programming language, or \emph{Pythonidae} family of snakes?). Even if a well-formed information need, it may be complex, and as such, many of the queries submitted to commercial web search engines today are ambiguous~\cite{sparckjones2007ambiguity}.

% How can a search engine tackle such ambiguity? As outlined by Santos et al.~\cite{santos2010query_reformulations_diversification}, the designers of search engines can tackle this problem in four different ways. Designers can either make their search engine: \emph{(i)} totally ignore the fact that the query may be ambiguous, instead treating the query as a representation of a well-defined information need; \emph{(ii)} infer the most plausible meaning for the underlying query, perhaps by examining historical search logs to determine the most popular inferred meaning; \emph{(iii)} explicitly prompt the user for feedback; or \emph{(iv)} \emph{diversify} the results that are returned to the searcher. The first three options are inherently risky; there is a chance that by inferring a meaning to an ambiguous query, the inferred meaning may be wrong. So too is the concept of prompting for explicit feedback; users may be unwilling to go the extra mile to provide this clarification~\cite{hearst2009_search} -- and indeed, this may create a loop where further clarification is required. As such, providing a more \emph{diverse} set of results for the ambiguous query would be a safe bet -- hopefully one of the meanings would satisfy the user's information need~\cite{agrawal2009diversification}.

% \begin{itemize}
%     \item{How, though, does this affect the behaviour of a searcher?}
%     \item{Previous user studies typically ask a user to examine content for x minutes, finding as many relevant documents as possible in that timeframe.}
%     \item{But the task undoubtedly changes user behaviour.}
%     \item{What if the overall goal is different? Rather than find as many as possible, what about find $x$ documents?}
%     \item{As such, in this study, we want to see what happens to the behaviour, performance and general experience of subjects when subjected to a search interface, tasked to find $x$ relevant items to a topic, diversified or not. This is also considered from the point of view from the search engine itself, where we include a diversification algorithm to potentially assist the searcher in identifying more diverse content.}
%     \item{\textbf{RQ1} How does search diversification affect a user's search behaviour and performance?}
%     \item{\textbf{RQ2} How is search behaviour and performance affected when the overall search task required diversification?}
%     \item{ARGH}
% \end{itemize}